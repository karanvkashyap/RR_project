% Options for packages loaded elsewhere
\PassOptionsToPackage{unicode}{hyperref}
\PassOptionsToPackage{hyphens}{url}
%
\documentclass[
]{article}
\usepackage{amsmath,amssymb}
\usepackage{lmodern}
\usepackage{iftex}
\ifPDFTeX
  \usepackage[T1]{fontenc}
  \usepackage[utf8]{inputenc}
  \usepackage{textcomp} % provide euro and other symbols
\else % if luatex or xetex
  \usepackage{unicode-math}
  \defaultfontfeatures{Scale=MatchLowercase}
  \defaultfontfeatures[\rmfamily]{Ligatures=TeX,Scale=1}
\fi
% Use upquote if available, for straight quotes in verbatim environments
\IfFileExists{upquote.sty}{\usepackage{upquote}}{}
\IfFileExists{microtype.sty}{% use microtype if available
  \usepackage[]{microtype}
  \UseMicrotypeSet[protrusion]{basicmath} % disable protrusion for tt fonts
}{}
\makeatletter
\@ifundefined{KOMAClassName}{% if non-KOMA class
  \IfFileExists{parskip.sty}{%
    \usepackage{parskip}
  }{% else
    \setlength{\parindent}{0pt}
    \setlength{\parskip}{6pt plus 2pt minus 1pt}}
}{% if KOMA class
  \KOMAoptions{parskip=half}}
\makeatother
\usepackage{xcolor}
\IfFileExists{xurl.sty}{\usepackage{xurl}}{} % add URL line breaks if available
\IfFileExists{bookmark.sty}{\usepackage{bookmark}}{\usepackage{hyperref}}
\hypersetup{
  pdftitle={Time Series Minimum Wage Studies Meta-Analysis - Report},
  pdfauthor={Gabriele Stella, Weida Pan, Karan Kashyap},
  hidelinks,
  pdfcreator={LaTeX via pandoc}}
\urlstyle{same} % disable monospaced font for URLs
\usepackage[margin=1in]{geometry}
\usepackage{graphicx}
\makeatletter
\def\maxwidth{\ifdim\Gin@nat@width>\linewidth\linewidth\else\Gin@nat@width\fi}
\def\maxheight{\ifdim\Gin@nat@height>\textheight\textheight\else\Gin@nat@height\fi}
\makeatother
% Scale images if necessary, so that they will not overflow the page
% margins by default, and it is still possible to overwrite the defaults
% using explicit options in \includegraphics[width, height, ...]{}
\setkeys{Gin}{width=\maxwidth,height=\maxheight,keepaspectratio}
% Set default figure placement to htbp
\makeatletter
\def\fps@figure{htbp}
\makeatother
\setlength{\emergencystretch}{3em} % prevent overfull lines
\providecommand{\tightlist}{%
  \setlength{\itemsep}{0pt}\setlength{\parskip}{0pt}}
\setcounter{secnumdepth}{-\maxdimen} % remove section numbering
\usepackage{booktabs}
\usepackage{longtable}
\usepackage{array}
\usepackage{multirow}
\usepackage{wrapfig}
\usepackage{float}
\usepackage{colortbl}
\usepackage{pdflscape}
\usepackage{tabu}
\usepackage{threeparttable}
\usepackage{threeparttablex}
\usepackage[normalem]{ulem}
\usepackage{makecell}
\usepackage{xcolor}
\ifLuaTeX
  \usepackage{selnolig}  % disable illegal ligatures
\fi

\title{Time Series Minimum Wage Studies Meta-Analysis - Report}
\author{Gabriele Stella, Weida Pan, Karan Kashyap}
\date{June, 2022}

\begin{document}
\maketitle

{
\setcounter{tocdepth}{2}
\tableofcontents
}
\begin{verbatim}
## Warning: package 'kableExtra' was built under R version 4.1.3
\end{verbatim}

\hypertarget{card-and-kruegers-meta-analysis}{%
\section{\texorpdfstring{\emph{Card and Krueger}'s
Meta-Analysis}{Card and Krueger's Meta-Analysis}}\label{card-and-kruegers-meta-analysis}}

A Meta-Analysis, is directed to use the main results about the same
argument and verify is the conclusion works also in aggregate, thanks to
this analysis we can also test a series of publication bias.

In fact, in this works \emph{card and Krueger} went to test what they
called ``\emph{one of the best-known predictions of standard economic
theory}'': \textbf{The effect of the minimum wage on the employment.}

Since the 1970 economics' literature argue about this effect and it's
well know that an increase in the minimum wage leads to increase the
unemployment rate on the weakest part of the employment force
(\emph{teenage one}) This prediction says that an increase of a 10\% in
the federal minimum wage leads to a reduction between one and three
percent in a teenage employment.

\emph{David Card \& Alan B. Krueger} paper is a meta-analysis - - sort
of study of publication bias in the minimum wage literature basic idea
is given a certain effect size and you have a bigger sample for the same
research design and the same effect you just have more precise estimates
therefore T-statistics should be larger. This was the starting point of
card and krueger and whether this actually holds in not in a literature
of 3-4 studies but actually in 15-20 studies by the time they do their
review in mid-90's.

\begin{itemize}
\tightlist
\item
  Critical Policy Questions '' What happens to employment when the min.
  wage goes up(i.e effect of min. wage on employement)
\end{itemize}

Despite that, * Card and Krueger* figure out that also if the past
studies are now accepted and seen as credible, the prediction of this
effect in a number of recent studies based on cross-sectional comparison
have estimated negligible or even marginally positive employment
effects.

In their meta-analysis \emph{Card and Krueger} grouped a group of
fifteen studies since 1970 to 1992. What they expected was to find,
since the conclusions was always more or less the same, a positive
correlation between the sample size of the studies and the t-ratio of
their coefficient.

Just about it, it's important to say that recent studies has more
observation than the previous one also because of historical reason.

Their analysis builds on the hypothesis that the literature contains an
unbiased sample of the coefficients and t-ratio due to the different on
the sample size as argue before.

Otherwise, their findings was difficult to reconcile with their null
hypothesis, so they explain that this result may be due to the tendency
for statistically significant results to be overrepresented in the
published literature, the author's to see publication their studying
could (voluntarily or not) have been affected by publication bias and/or
specification searching.

\hypertarget{our-implementation}{%
\section{Our implementation}\label{our-implementation}}

Our task it's to implement the \emph{Card and Krueger}'s dataframe we
searched and found three more studies:

\begin{itemize}
\tightlist
\item
  The Impact of the National Minimum Wage on Employment (2011)

  \begin{itemize}
  \tightlist
  \item
    \emph{Jinlan Ni} , \emph{Guangxin Wang} and \emph{Xianguo Yao}
  \end{itemize}
\item
  The impact of minimum wage on employment in Poland (2012)

  \begin{itemize}
  \tightlist
  \item
    \emph{Aleksandra Majchrowska} and \emph{Zbigniew Żółkiewski}
  \end{itemize}
\item
  Impact of Minimum Wages on Employment (2015)

  \begin{itemize}
  \tightlist
  \item
    \emph{Eewley et al.}
  \end{itemize}
\end{itemize}

\hypertarget{data-summarize}{%
\subsection{Data Summarize}\label{data-summarize}}

\hypertarget{card-and-kruegers-data.}{%
\subsubsection{Card and Krueger's Data.}\label{card-and-kruegers-data.}}

\begin{table}
\centering\begingroup\fontsize{11}{13}\selectfont

\begin{tabular}{rllrrrrrrrrrrr}
\toprule
id\_num & implemented & author & year & t\_stat & df & coef & teen\_subsample & log\_spec & no\_exp\_var & autoreg\_correction & error & sqrt\_df & l\_sqrt\_df\\
\midrule
1 & No & Kaitz & 1970 & 2.30 & 49 & 0.098 & NA & 0 & 10 & 0 & 0.0426087 & 7.000000 & 1.945910\\
2 & No & Mincer & 1976 & 2.41 & 58 & 0.231 & NA & 0 & 5 & 1 & 0.0958506 & 7.615773 & 2.030222\\
3 & No & Gramlich & 1976 & 1.41 & 106 & 0.094 & NA & 1 & 17 & 1 & 0.0666667 & 10.295630 & 2.331720\\
4 & No & Welch & 1976 & 2.22 & 53 & 0.178 & NA & 1 & 6 & 0 & 0.0801802 & 7.280110 & 1.985146\\
5 & No & Ragan & 1977 & 1.52 & 31 & 0.065 & NA & 1 & 8 & 1 & 0.0427632 & 5.567764 & 1.716994\\
\addlinespace
6 & No & Wachter and Kim & 1979 & 2.17 & 56 & 0.252 & NA & 1 & 11 & 0 & 0.1161290 & 7.483315 & 2.012676\\
7 & No & Iden & 1980 & 4.43 & 93 & 0.226 & NA & 0 & 10 & 1 & 0.0510158 & 9.643651 & 2.266300\\
8 & No & Ragan & 1981 & 1.70 & 54 & 0.052 & NA & 1 & 9 & 1 & 0.0305882 & 7.348469 & 1.994492\\
9 & No & Abowd and Killingsworth & 1981 & 1.04 & 95 & 0.213 & NA & 1 & 8 & 0 & 0.2048077 & 9.746794 & 2.276938\\
10 & No & Betsey and Dunson & 1981 & 2.12 & 93 & 0.139 & NA & 0 & 10 & 1 & 0.0655660 & 9.643651 & 2.266300\\
\addlinespace
11 & No & Brown & 1983 & 1.92 & 92 & 0.096 & NA & 1 & 11 & 0 & 0.0500000 & 9.591663 & 2.260894\\
12 & No & Hammermesh & 1981 & 1.63 & 94 & 0.121 & NA & 1 & 5 & 1 & 0.0742331 & 9.695360 & 2.271647\\
13 & No & Solon & 1985 & 2.78 & 86 & 0.098 & NA & 1 & 17 & 1 & 0.0352518 & 9.273618 & 2.227174\\
14 & No & Wellington & 1991 & 1.41 & 114 & 0.066 & NA & 1 & 17 & 1 & 0.0468085 & 10.677078 & 2.368099\\
15 & No & Klerman & 1992 & 0.45 & 123 & 0.052 & NA & 1 & 5 & 1 & 0.1155556 & 11.090537 & 2.406092\\
\bottomrule
\end{tabular}
\endgroup{}
\end{table}

In the next session you can see the original Dataframe, the implemented
data frame and the full dataframe aggregate; follow by a legend for the
variable. Then we're going to proceed with three more chapter, in each
chapter you can see the model or graph estimated by \emph{Card and
Krueger} follow by the same study but on our implemented dataframe.

\hypertarget{our-implementation.}{%
\subsubsection{Our implementation.}\label{our-implementation.}}

\begin{table}
\centering\begingroup\fontsize{11}{13}\selectfont

\begin{tabular}{lrllrrrrrrrrrrr}
\toprule
  & id\_num & implemented & author & year & t\_stat & df & coef & teen\_subsample & log\_spec & no\_exp\_var & autoreg\_correction & error & sqrt\_df & l\_sqrt\_df\\
\midrule
16 & 16 & Yes & Majchrowska and Zo´lkiewski & 2012 & 2.15 & 157 & 0.270 & 1 & 1 & 8 & 1 & 0.1255814 & 12.52996 & 2.528123\\
17 & 17 & Yes & Ni ,Wang \& Yao & 2011 & 1.97 & 105 & 0.098 & 1 & 1 & 6 & 1 & 0.0497462 & 10.24695 & 2.326980\\
18 & 18 & Yes & Bewley et al. & 2015 & 1.46 & 128 & 0.072 & 1 & 1 & 10 & 1 & 0.0493151 & 11.31371 & 2.426015\\
\bottomrule
\end{tabular}
\endgroup{}
\end{table}

Our result are based on sample size really huge, but their t-stat
doesn't seem highly significant and that's in line with the \emph{Card
and Krueger} prediction.

In the next session you can see the original Dataframe, the implemented
data frame and the full dataframe aggregate; follow by a legend for the
variable. Then we're going to proceed with three more chapter, in each
chapter you can see the model or graph estimated by \emph{Card and
Krueger} follow by the same study but on our implemented dataframe.

\hypertarget{meta-analysis-data-frame}{%
\subsubsection{Meta analysis Data
Frame}\label{meta-analysis-data-frame}}

\begin{table}
\centering\begingroup\fontsize{11}{13}\selectfont

\begin{tabular}{rllrrrrrrrrrrr}
\toprule
id\_num & implemented & author & year & t\_stat & df & coef & teen\_subsample & log\_spec & no\_exp\_var & autoreg\_correction & error & sqrt\_df & l\_sqrt\_df\\
\midrule
1 & No & Kaitz & 1970 & 2.30 & 49 & 0.098 & NA & 0 & 10 & 0 & 0.0426087 & 7.000000 & 1.945910\\
2 & No & Mincer & 1976 & 2.41 & 58 & 0.231 & NA & 0 & 5 & 1 & 0.0958506 & 7.615773 & 2.030222\\
3 & No & Gramlich & 1976 & 1.41 & 106 & 0.094 & NA & 1 & 17 & 1 & 0.0666667 & 10.295630 & 2.331720\\
4 & No & Welch & 1976 & 2.22 & 53 & 0.178 & NA & 1 & 6 & 0 & 0.0801802 & 7.280110 & 1.985146\\
5 & No & Ragan & 1977 & 1.52 & 31 & 0.065 & NA & 1 & 8 & 1 & 0.0427632 & 5.567764 & 1.716994\\
\addlinespace
6 & No & Wachter and Kim & 1979 & 2.17 & 56 & 0.252 & NA & 1 & 11 & 0 & 0.1161290 & 7.483315 & 2.012676\\
7 & No & Iden & 1980 & 4.43 & 93 & 0.226 & NA & 0 & 10 & 1 & 0.0510158 & 9.643651 & 2.266300\\
8 & No & Ragan & 1981 & 1.70 & 54 & 0.052 & NA & 1 & 9 & 1 & 0.0305882 & 7.348469 & 1.994492\\
9 & No & Abowd and Killingsworth & 1981 & 1.04 & 95 & 0.213 & NA & 1 & 8 & 0 & 0.2048077 & 9.746794 & 2.276938\\
10 & No & Betsey and Dunson & 1981 & 2.12 & 93 & 0.139 & NA & 0 & 10 & 1 & 0.0655660 & 9.643651 & 2.266300\\
\addlinespace
11 & No & Brown & 1983 & 1.92 & 92 & 0.096 & NA & 1 & 11 & 0 & 0.0500000 & 9.591663 & 2.260894\\
12 & No & Hammermesh & 1981 & 1.63 & 94 & 0.121 & NA & 1 & 5 & 1 & 0.0742331 & 9.695360 & 2.271647\\
13 & No & Solon & 1985 & 2.78 & 86 & 0.098 & NA & 1 & 17 & 1 & 0.0352518 & 9.273618 & 2.227174\\
14 & No & Wellington & 1991 & 1.41 & 114 & 0.066 & NA & 1 & 17 & 1 & 0.0468085 & 10.677078 & 2.368099\\
15 & No & Klerman & 1992 & 0.45 & 123 & 0.052 & NA & 1 & 5 & 1 & 0.1155556 & 11.090537 & 2.406092\\
\addlinespace
16 & Yes & Majchrowska and Zo´lkiewski & 2012 & 2.15 & 157 & 0.270 & 1 & 1 & 8 & 1 & 0.1255814 & 12.529964 & 2.528123\\
17 & Yes & Ni ,Wang \& Yao & 2011 & 1.97 & 105 & 0.098 & 1 & 1 & 6 & 1 & 0.0497462 & 10.246951 & 2.326980\\
18 & Yes & Bewley et al. & 2015 & 1.46 & 128 & 0.072 & 1 & 1 & 10 & 1 & 0.0493151 & 11.313709 & 2.426015\\
\bottomrule
\end{tabular}
\endgroup{}
\end{table}

In the next session you can see the original Dataframe, the implemented
data frame and the full dataframe aggregate; follow by a legend for the
variable. Then we're going to proceed with three more chapter, in each
chapter you can see the model or graph estimated by \emph{Card and
Krueger} follow by the same study but on our implemented dataframe.

\hypertarget{variable}{%
\subsection{variable}\label{variable}}

as we can see they used a set of 15 studies and we went to implement
with three studies most recent.

the variable that they use was 12:

\begin{itemize}
\tightlist
\item
  consists of the author(s),
\item
  date of publication (year),
\item
  t-statistic in absolute terms1 (t\_stat),
\item
  degrees of freedom (df),
\item
  coefficient of the minimum wage variable in absolute terms2 (coef),
\item
  number of explanatory variables of the model (no\_exp\_var),
\item
  the error of the model (error) {[}Computing log and sqrt\_log of
  error.{]}`
\item
  and three dummy varibale:

  \begin{itemize}
  \tightlist
  \item
    log\_spec

    \begin{itemize}
    \tightlist
    \item
      this variable is set to 1 when the author used a logarithmic
      specification for the model
    \end{itemize}
  \item
    autoreg\_correction

    \begin{itemize}
    \tightlist
    \item
      this is set to 1, when the autoregression correction was applied
    \end{itemize}
  \item
    teen\_subsample

    \begin{itemize}
    \tightlist
    \item
      this variable is set to one when the authors used a teen
      sumbsample, unfortunatly we didn't find enough information to fill
      the missing data.
    \end{itemize}
  \end{itemize}
\end{itemize}

\hypertarget{reproducing-part}{%
\section{2 Reproducing Part}\label{reproducing-part}}

the authors implemented this study focusing on three cases:

\hypertarget{relation-of-estimated-t-ratio-to-sample-size}{%
\subsection{2\_1 Relation of Estimated t-Ratio to Sample
Size}\label{relation-of-estimated-t-ratio-to-sample-size}}

\includegraphics{Report-_files/figure-latex/unnamed-chunk-5-1.pdf}

\hypertarget{descreptive-regression-models}{%
\subsection{2\_2 Descreptive Regression
Models}\label{descreptive-regression-models}}

\hypertarget{relation-of-estimated-employment-to-standard-error}{%
\subsection{2\_3 Relation of Estimated Employment to Standard
Error}\label{relation-of-estimated-employment-to-standard-error}}

\includegraphics{Report-_files/figure-latex/unnamed-chunk-6-1.pdf}

\hypertarget{implementation}{%
\section{3 Implementation}\label{implementation}}

Our assignment doesn't concern just the reproducing part, but now we
want to implement the same study implemented with our data.

\hypertarget{relation-of-estimated-t-ratio-to-sample-size--implemented}{%
\subsection{3\_1 Relation of Estimated t-Ratio to Sample Size-
Implemented}\label{relation-of-estimated-t-ratio-to-sample-size--implemented}}

\includegraphics{Report-_files/figure-latex/unnamed-chunk-7-1.pdf}

\hypertarget{descreptive-regression-models--implemented}{%
\subsection{3\_2 Descreptive Regression Models-
Implemented}\label{descreptive-regression-models--implemented}}

\hypertarget{relation-of-estimated-employment-to-standard-error--implemented}{%
\subsection{3\_3 Relation of Estimated Employment to Standard Error-
Implemented}\label{relation-of-estimated-employment-to-standard-error--implemented}}

\includegraphics{Report-_files/figure-latex/unnamed-chunk-8-1.pdf}

\hypertarget{conclusion}{%
\section{4 conclusion}\label{conclusion}}

\hypertarget{reference}{%
\section{reference}\label{reference}}

\end{document}
