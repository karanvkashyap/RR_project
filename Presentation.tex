% Options for packages loaded elsewhere
\PassOptionsToPackage{unicode}{hyperref}
\PassOptionsToPackage{hyphens}{url}
%
\documentclass[
  9 pt,
  ignorenonframetext,
]{beamer}
\usepackage{pgfpages}
\setbeamertemplate{caption}[numbered]
\setbeamertemplate{caption label separator}{: }
\setbeamercolor{caption name}{fg=normal text.fg}
\beamertemplatenavigationsymbolsempty
% Prevent slide breaks in the middle of a paragraph
\widowpenalties 1 10000
\raggedbottom
\setbeamertemplate{part page}{
  \centering
  \begin{beamercolorbox}[sep=16pt,center]{part title}
    \usebeamerfont{part title}\insertpart\par
  \end{beamercolorbox}
}
\setbeamertemplate{section page}{
  \centering
  \begin{beamercolorbox}[sep=12pt,center]{part title}
    \usebeamerfont{section title}\insertsection\par
  \end{beamercolorbox}
}
\setbeamertemplate{subsection page}{
  \centering
  \begin{beamercolorbox}[sep=8pt,center]{part title}
    \usebeamerfont{subsection title}\insertsubsection\par
  \end{beamercolorbox}
}
\AtBeginPart{
  \frame{\partpage}
}
\AtBeginSection{
  \ifbibliography
  \else
    \frame{\sectionpage}
  \fi
}
\AtBeginSubsection{
  \frame{\subsectionpage}
}
\usepackage{amsmath,amssymb}
\usepackage{lmodern}
\usepackage{iftex}
\ifPDFTeX
  \usepackage[T1]{fontenc}
  \usepackage[utf8]{inputenc}
  \usepackage{textcomp} % provide euro and other symbols
\else % if luatex or xetex
  \usepackage{unicode-math}
  \defaultfontfeatures{Scale=MatchLowercase}
  \defaultfontfeatures[\rmfamily]{Ligatures=TeX,Scale=1}
\fi
\usetheme[]{Warsaw}
\usecolortheme{lily}
% Use upquote if available, for straight quotes in verbatim environments
\IfFileExists{upquote.sty}{\usepackage{upquote}}{}
\IfFileExists{microtype.sty}{% use microtype if available
  \usepackage[]{microtype}
  \UseMicrotypeSet[protrusion]{basicmath} % disable protrusion for tt fonts
}{}
\makeatletter
\@ifundefined{KOMAClassName}{% if non-KOMA class
  \IfFileExists{parskip.sty}{%
    \usepackage{parskip}
  }{% else
    \setlength{\parindent}{0pt}
    \setlength{\parskip}{6pt plus 2pt minus 1pt}}
}{% if KOMA class
  \KOMAoptions{parskip=half}}
\makeatother
\usepackage{xcolor}
\IfFileExists{xurl.sty}{\usepackage{xurl}}{} % add URL line breaks if available
\IfFileExists{bookmark.sty}{\usepackage{bookmark}}{\usepackage{hyperref}}
\hypersetup{
  pdftitle={Time Series Minimum Wage Studies Meta-Analysis - Presentation},
  pdfauthor={Gabriele Stella, Weida Pan, Karan Kashyap},
  hidelinks,
  pdfcreator={LaTeX via pandoc}}
\urlstyle{same} % disable monospaced font for URLs
\newif\ifbibliography
\usepackage{color}
\usepackage{fancyvrb}
\newcommand{\VerbBar}{|}
\newcommand{\VERB}{\Verb[commandchars=\\\{\}]}
\DefineVerbatimEnvironment{Highlighting}{Verbatim}{commandchars=\\\{\}}
% Add ',fontsize=\small' for more characters per line
\usepackage{framed}
\definecolor{shadecolor}{RGB}{248,248,248}
\newenvironment{Shaded}{\begin{snugshade}}{\end{snugshade}}
\newcommand{\AlertTok}[1]{\textcolor[rgb]{0.94,0.16,0.16}{#1}}
\newcommand{\AnnotationTok}[1]{\textcolor[rgb]{0.56,0.35,0.01}{\textbf{\textit{#1}}}}
\newcommand{\AttributeTok}[1]{\textcolor[rgb]{0.77,0.63,0.00}{#1}}
\newcommand{\BaseNTok}[1]{\textcolor[rgb]{0.00,0.00,0.81}{#1}}
\newcommand{\BuiltInTok}[1]{#1}
\newcommand{\CharTok}[1]{\textcolor[rgb]{0.31,0.60,0.02}{#1}}
\newcommand{\CommentTok}[1]{\textcolor[rgb]{0.56,0.35,0.01}{\textit{#1}}}
\newcommand{\CommentVarTok}[1]{\textcolor[rgb]{0.56,0.35,0.01}{\textbf{\textit{#1}}}}
\newcommand{\ConstantTok}[1]{\textcolor[rgb]{0.00,0.00,0.00}{#1}}
\newcommand{\ControlFlowTok}[1]{\textcolor[rgb]{0.13,0.29,0.53}{\textbf{#1}}}
\newcommand{\DataTypeTok}[1]{\textcolor[rgb]{0.13,0.29,0.53}{#1}}
\newcommand{\DecValTok}[1]{\textcolor[rgb]{0.00,0.00,0.81}{#1}}
\newcommand{\DocumentationTok}[1]{\textcolor[rgb]{0.56,0.35,0.01}{\textbf{\textit{#1}}}}
\newcommand{\ErrorTok}[1]{\textcolor[rgb]{0.64,0.00,0.00}{\textbf{#1}}}
\newcommand{\ExtensionTok}[1]{#1}
\newcommand{\FloatTok}[1]{\textcolor[rgb]{0.00,0.00,0.81}{#1}}
\newcommand{\FunctionTok}[1]{\textcolor[rgb]{0.00,0.00,0.00}{#1}}
\newcommand{\ImportTok}[1]{#1}
\newcommand{\InformationTok}[1]{\textcolor[rgb]{0.56,0.35,0.01}{\textbf{\textit{#1}}}}
\newcommand{\KeywordTok}[1]{\textcolor[rgb]{0.13,0.29,0.53}{\textbf{#1}}}
\newcommand{\NormalTok}[1]{#1}
\newcommand{\OperatorTok}[1]{\textcolor[rgb]{0.81,0.36,0.00}{\textbf{#1}}}
\newcommand{\OtherTok}[1]{\textcolor[rgb]{0.56,0.35,0.01}{#1}}
\newcommand{\PreprocessorTok}[1]{\textcolor[rgb]{0.56,0.35,0.01}{\textit{#1}}}
\newcommand{\RegionMarkerTok}[1]{#1}
\newcommand{\SpecialCharTok}[1]{\textcolor[rgb]{0.00,0.00,0.00}{#1}}
\newcommand{\SpecialStringTok}[1]{\textcolor[rgb]{0.31,0.60,0.02}{#1}}
\newcommand{\StringTok}[1]{\textcolor[rgb]{0.31,0.60,0.02}{#1}}
\newcommand{\VariableTok}[1]{\textcolor[rgb]{0.00,0.00,0.00}{#1}}
\newcommand{\VerbatimStringTok}[1]{\textcolor[rgb]{0.31,0.60,0.02}{#1}}
\newcommand{\WarningTok}[1]{\textcolor[rgb]{0.56,0.35,0.01}{\textbf{\textit{#1}}}}
\usepackage{graphicx}
\makeatletter
\def\maxwidth{\ifdim\Gin@nat@width>\linewidth\linewidth\else\Gin@nat@width\fi}
\def\maxheight{\ifdim\Gin@nat@height>\textheight\textheight\else\Gin@nat@height\fi}
\makeatother
% Scale images if necessary, so that they will not overflow the page
% margins by default, and it is still possible to overwrite the defaults
% using explicit options in \includegraphics[width, height, ...]{}
\setkeys{Gin}{width=\maxwidth,height=\maxheight,keepaspectratio}
% Set default figure placement to htbp
\makeatletter
\def\fps@figure{htbp}
\makeatother
\setlength{\emergencystretch}{3em} % prevent overfull lines
\providecommand{\tightlist}{%
  \setlength{\itemsep}{0pt}\setlength{\parskip}{0pt}}
\setcounter{secnumdepth}{-\maxdimen} % remove section numbering
\ifLuaTeX
  \usepackage{selnolig}  % disable illegal ligatures
\fi

\title{Time Series Minimum Wage Studies Meta-Analysis - Presentation}
\author{Gabriele Stella, Weida Pan, Karan Kashyap}
\date{June, 2022}

\begin{document}
\frame{\titlepage}

\begin{frame}[fragile]
\begin{Shaded}
\begin{Highlighting}[]
\FunctionTok{load}\NormalTok{(}\StringTok{\textquotesingle{}Data/Environment.Rdata\textquotesingle{}}\NormalTok{)}
\end{Highlighting}
\end{Shaded}
\end{frame}

\begin{frame}[fragile]{Introduction}
\protect\hypertarget{introduction}{}
\begin{itemize}
\item
  our project is based on the paper written by \emph{David Card \& Alan
  B. Krueger} in 1995 regarding research of the best-known predictions
  of standard economic theory that an increase in the minimum wage will
  lower employment of low-wage workers. Their result indicates that an
  increase of 10\% in the minimum wage as a result a reduction between
  1-3 \% in teenage employment.
\item
  The main idea of our project is to reproduce their research by add our
  dataframe via Meta-Analysis
\item
  In this work, we're going to implement the previous meta analysis with
  3 most recent studies.
\item
  requirements to re-run the obtained results are:
\end{itemize}

\texttt{R\ version\ 4.1.3}, \texttt{rmarkdown\ 2.14},
\texttt{tidyverse\ 1.3.1}, \texttt{stargazer\ 5.2.3},
\texttt{rcompanion\ 2.4.15}.
\end{frame}

\begin{frame}{Meta-Analysis}
\protect\hypertarget{meta-analysis}{}
\begin{itemize}
\item
  A meta-analysis is a statistical analysis that combines the results of
  multiple scientific studies. Meta-analyses can be performed when there
  are multiple scientific studies addressing the same question, with
  each individual study reporting measurements that are expected to have
  some degree of error.
\item
  The aim then is to use approaches from statistics to derive a pooled
  estimate closest to the unknown common truth based on how this error
  is perceived. Meta-analytic results are considered the most
  trustworthy source of evidence by the evidence-based medicine
  literature
\end{itemize}
\end{frame}

\begin{frame}{Data (1/3)}
\protect\hypertarget{data-13}{}
\begin{itemize}
\item
  the professor give us the Data of the \emph{Card and krueger's
  Meta-Ananlysis}, but in that table missing three variable:

  \begin{itemize}
  \item
    information about teenager sub-sample
  \item
    square root of degree of freedom
  \item
    logarithm of square root of degree of freedom
  \end{itemize}
\item
  we would be able to calculate the last two, but also reading the paper
  of the others author we couldn't find enough information about
  teenager sub-sample.
\item
  The original study includes 15 observations of 14 variables by year
  1992.
\end{itemize}
\end{frame}

\begin{frame}{Data (2/3)}
\protect\hypertarget{data-23}{}
\begin{itemize}
\item
  our implementation concern:

  \begin{itemize}
  \item
    J\emph{inlan Ni , Guangxin Wang \& Xianguo Yao}, ``Impact of Minimum
    Wages on Employment'', \textbf{2011}
  \item
    \emph{Majchrowska and Zolkiew}, ``The Impact of the National Minimum
    Wage on Employment'', \textbf{2012}
  \item
    \emph{Bewley et al.}, ``The Impact of the National Minimum Wage on
    Employment'', \textbf{2015}
  \end{itemize}
\item
  Below we will conduct two regression models which present by graph and
  table, first we'll use just the previous fifteen studies to reproduce
  the Card and Krueger's M-A, then we'll implement with our data and
  we'll see if the study keeps its validity
\end{itemize}
\end{frame}

\begin{frame}[fragile]{Data (3/3)}
\protect\hypertarget{data-33}{}
\begin{block}{\textbf{t\_stat}}
\protect\hypertarget{t_stat}{}
\begin{verbatim}
##    Min. 1st Qu.  Median    Mean 3rd Qu.    Max. 
##   0.450   1.475   1.945   1.949   2.208   4.430
\end{verbatim}
\end{block}

\begin{block}{\textbf{coef}}
\protect\hypertarget{coef}{}
\begin{verbatim}
##    Min. 1st Qu.  Median    Mean 3rd Qu.    Max. 
##  0.0520  0.0775  0.0980  0.1345  0.2042  0.2700
\end{verbatim}
\end{block}

\begin{block}{\textbf{df}}
\protect\hypertarget{df}{}
\begin{verbatim}
##    Min. 1st Qu.  Median    Mean 3rd Qu.    Max. 
##   31.00   56.50   93.00   88.17  105.75  157.00
\end{verbatim}
\end{block}
\end{frame}

\begin{frame}{procedure}
\protect\hypertarget{procedure}{}
\begin{itemize}
\item
  Now we're going to show our reproduction of the \emph{Card and
  Krueger} study and then we'll compare their result with the same study
  implemented with our implementation.
\item
  We followed step by step their analysis, the first figure shows us the
  correlation between t\_stat and logarithm of square root of degree of
  freedom.
\item
  As you could see our result are consistent with the previuos
  Meta-Analysis.
\end{itemize}
\end{frame}

\begin{frame}{Relation of Estimated t-Ratio to Sample Size (1/2)}
\protect\hypertarget{relation-of-estimated-t-ratio-to-sample-size-12}{}
-Graph Explanation: In the below graph we can see that with big
samples(12,3,4,9,15) given the same effect size has T-Statistics less
than 2, whereas smaller sample size somehow have just above 2(1,4,6,2).

\includegraphics{Presentation_files/figure-beamer/unnamed-chunk-5-1.pdf}
\end{frame}

\begin{frame}{Relation of Estimated t-Ratio to Sample Size (2/2)}
\protect\hypertarget{relation-of-estimated-t-ratio-to-sample-size-22}{}
Graph Explanation: In the below graph we can see that the newly added
data with bigger sample given the same effect has T-Statistics very
close to 2.(16,17,18). It means it has more precises estimates.

\includegraphics{Presentation_files/figure-beamer/unnamed-chunk-6-1.pdf}
\end{frame}

\begin{frame}{Procedure}
\protect\hypertarget{procedure-1}{}
\begin{itemize}
\item
  In order to control the potential impact of other research
  characteristics on this dependence, we performed regressions with the
  logarithm of absolute t-ratio as the dependent variable selected
  independent variables were the logarithm of the square root of degrees
  of freedom, a number of explanatory variables in given research, and
  binary variables indicating whether the model specification in given
  research was logarithmic and if the autoregressive correction was
  implemented.
\item
  As we could see the graph shows a negative relationship between the t
  ratios and the degrees of freedom. The coefficient on the square root
  of the degrees of freedom is quite far from 1, its theoretical
  expectation. The inclusion of additional explanatory variables does
  not change the sign of the coefficient or reduce its effect.
\item
  Next, we will present the reproduced regressions in Table which gives
  us more information about the correlation. Furthermore, due to the
  fact the study number 7 is considered as biased data set, therefore we
  will also conduct table to analysis the result which excludes study
  number 7.
\end{itemize}
\end{frame}

\begin{frame}{Descreptive Regression Models (1/2)}
\protect\hypertarget{descreptive-regression-models-12}{}
\begin{itemize}
\tightlist
\item
  The striking pattern to notice that most of this studies have
  T-statistics just above 2 but in this case when they do regression the
  log t-stat on log degree of freedom instead of getting a slope of 1
  they get a negative slope -0.94, This is deeply problematic if it was
  zero then also it was problematic but now it is negative, as seen in
  Relation of Estimated t-Ratio to Sample Size the previous graph
  cluster with big sample size given the same effect size which should
  have bigger t-statistics is actually getting a zero.
\end{itemize}

\begin{table}[!htbp] \centering 

  \label{} 
\footnotesize 
\begin{tabular}{@{\extracolsep{5pt}}lccc} 
\\[-1.8ex]\hline 
\hline \\[-1.8ex] 
 & \multicolumn{3}{c}{\textit{Dependent variable:}} \\ 
\cline{2-4} 
\\[-1.8ex] & \multicolumn{3}{c}{log(t\_stat)} \\ 
\\[-1.8ex] & (1) & (2) & (3)\\ 
\hline \\[-1.8ex] 
 l\_sqrt\_df & $-$0.81 & $-$0.64 & $-$0.94 \\ 
  & (0.69) & (0.66) & (0.62) \\ 
  autoreg\_correction &  & $-$0.07 & $-$0.11 \\ 
  &  & (0.27) & (0.24) \\ 
  log\_spec &  & $-$0.55$^{*}$ & $-$0.63$^{**}$ \\ 
  &  & (0.28) & (0.26) \\ 
  no\_exp\_var &  &  & 0.05$^{*}$ \\ 
  &  &  & (0.03) \\ 
  Constant & 2.31 & 2.41 & 2.61$^{*}$ \\ 
  & (1.49) & (1.40) & (1.27) \\ 
 \hline \\[-1.8ex] 
Observations & 15 & 15 & 15 \\ 
R$^{2}$ & 0.10 & 0.33 & 0.50 \\ 
Adjusted R$^{2}$ & 0.03 & 0.15 & 0.30 \\ 
Residual Std. Error & 0.50 (df = 13) & 0.47 (df = 11) & 0.43 (df = 10) \\ 
F Statistic & 1.37 (df = 1; 13) & 1.83 (df = 3; 11) & 2.51 (df = 4; 10) \\ 
\hline 
\hline \\[-1.8ex] 
\textit{Note:}  & \multicolumn{3}{r}{$^{*}$p$<$0.1; $^{**}$p$<$0.05; $^{***}$p$<$0.01} \\ 
\end{tabular} 
\end{table}
\end{frame}

\begin{frame}{Descreptive Regression Models (2/2)}
\protect\hypertarget{descreptive-regression-models-22}{}
\begin{itemize}
\tightlist
\item
  The striking pattern to notice that when we added 3 new variables
  16,17,18 and did regression the log t-stat on log degree of freedom
  instead of getting a slope of 1 we got a negative slope -0.40 which is
  better than -0.94.
\end{itemize}

\begin{table}[!htbp] \centering 

  \label{} 
\footnotesize 
\begin{tabular}{@{\extracolsep{5pt}}lccc} 
\\[-1.8ex]\hline 
\hline \\[-1.8ex] 
 & \multicolumn{3}{c}{\textit{Dependent variable:}} \\ 
\cline{2-4} 
\\[-1.8ex] & \multicolumn{3}{c}{log(t\_stat)} \\ 
\\[-1.8ex] & (1) & (2) & (3)\\ 
\hline \\[-1.8ex] 
 l\_sqrt\_df & $-$0.54 & $-$0.32 & $-$0.40 \\ 
  & (0.55) & (0.56) & (0.56) \\ 
  autoreg\_correction &  & $-$0.03 & $-$0.03 \\ 
  &  & (0.25) & (0.25) \\ 
  log\_spec &  & $-$0.50$^{*}$ & $-$0.53$^{*}$ \\ 
  &  & (0.26) & (0.26) \\ 
  no\_exp\_var &  &  & 0.03 \\ 
  &  &  & (0.03) \\ 
  Constant & 1.76 & 1.68 & 1.59 \\ 
  & (1.21) & (1.17) & (1.15) \\ 
 \hline \\[-1.8ex] 
Observations & 18 & 18 & 18 \\ 
R$^{2}$ & 0.06 & 0.26 & 0.33 \\ 
Adjusted R$^{2}$ & $-$0.001 & 0.10 & 0.12 \\ 
Residual Std. Error & 0.47 (df = 16) & 0.44 (df = 14) & 0.44 (df = 13) \\ 
F Statistic & 0.98 (df = 1; 16) & 1.60 (df = 3; 14) & 1.58 (df = 4; 13) \\ 
\hline 
\hline \\[-1.8ex] 
\textit{Note:}  & \multicolumn{3}{r}{$^{*}$p$<$0.1; $^{**}$p$<$0.05; $^{***}$p$<$0.01} \\ 
\end{tabular} 
\end{table}
\end{frame}

\begin{frame}{is it t-statistics always 2?}
\protect\hypertarget{is-it-t-statistics-always-2}{}
-Another sort of meta-analysis compares the size of each study's
coefficient to its estimated standard error. If the same statistical
model remains true throughout time, there should be no systematic link
between the estimated coefficients and their standard errors. However,
if publication bias causes studies with t ratios greater than 2 to be
published, we would expect to discover a positive association between
the magnitude of the estimated employment effect and its standard
errors. Assume that journals have a rule of only publishing papers with
t ratios greater than 2, and that authors adjust their specifications
(by shifting functional forms, changing the list of included variables,
and so on) until they reach such a t ratio. Therefore we find many
published paper with t-ratios equal to or just above 2.
\end{frame}

\begin{frame}[fragile]{Histogram}
\protect\hypertarget{histogram}{}
In statistics, the ``normal distribution'' is the most widely used
distribution. A normal distribution has a histogram that is bell-shaped
and symmetric around the mean. Kurtosis and skewness are words used to
characterize deviations from normality. In the below graph we can see
the positive skew this means that the right side's tail is longer than
the left side's tail, and the majority of the values are to the somewhat
left of the mean. This is mostly due to the fact that there is a very
high upper restriction on intake but no lower limit.

\begin{verbatim}
## Warning: package 'rcompanion' was built under R version 4.1.3
\end{verbatim}

\includegraphics{Presentation_files/figure-beamer/unnamed-chunk-9-1.pdf}
\end{frame}

\begin{frame}{Procedure}
\protect\hypertarget{procedure-2}{}
\begin{itemize}
\item
  Now we're going to plot absolute value of the minimum-wage effect
  against their standard error.
\item
  This relation needs to looking these correlation, but also to see if
  the coefficients estimatedvare effectively 2 times their standard
  error (\emph{to see these we'll implement our graphs with a line that
  it's 2 times the standard error})
\end{itemize}
\end{frame}

\begin{frame}{Relation of Estimated Employment to Standard Error (1/2)}
\protect\hypertarget{relation-of-estimated-employment-to-standard-error-12}{}
\begin{itemize}
\tightlist
\item
  Graph Explanation - This is another way of looking at the data where
  we have standard error of estimate and estimated elasticity effect,
  The graph plot out 2 times the standard error and you can see the
  incredible clustering of studies right above 2 times of the standard
  error (13,1,10,4,2,6). This is very unnatural, The most concerning
  thing to observe here is that studies with less standard error
  (13,1,11,8,14,5) are well powered studies. Probably this are most
  reliable studies.
\end{itemize}

\includegraphics{Presentation_files/figure-beamer/unnamed-chunk-10-1.pdf}
\end{frame}

\begin{frame}{Relation of Estimated Employment to Standard Error (2/2)}
\protect\hypertarget{relation-of-estimated-employment-to-standard-error-22}{}
\begin{itemize}
\tightlist
\item
  When we added new studies we can observe that (17,18) have less
  standard errror where has 16 have high standard error. Thus we can
  assume with this that studies (17,18) are more reliable then study 16.
\end{itemize}

\includegraphics{Presentation_files/figure-beamer/unnamed-chunk-11-1.pdf}
\end{frame}

\begin{frame}{conclusion}
\protect\hypertarget{conclusion}{}
Our findings are consistent with the original meta-analysis by Card and
Krueger (1995), who argue that these countrintuitive results are caused
by the fact that, when the available data are not too large, the norms
used by researchers make them were able to find a significant negative
correlation between minimum wages and their expected youth employment.
However, due to its fragility, the significance decreases as more data
is used. Our supplementary paper confirms that this trend in research
bias is critical and still exists, with lower t-ratios for papers
published after the 15 papers studied by Card and Kruger, even with
larger data samples.

In conclusion, there is strong evidence that research is often biased.
Lack of self-awareness and excessive trust in existing literature can be
misleading.
\end{frame}

\begin{frame}{ended}
\protect\hypertarget{ended}{}
\centering  Thank You
\end{frame}

\end{document}
